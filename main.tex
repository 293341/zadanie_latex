\documentclass[11pt, a4paper]{report}

% ---------------------------------------------------------
% PAKIETY (PREAMBUŁA)
% ---------------------------------------------------------
\usepackage[polish]{babel}    % Język polski (dzielenie wyrazów, nazwy rozdziałów)
\usepackage[utf8]{inputenc}   % Kodowanie znaków
\usepackage[T1]{fontenc}      % Kodowanie fontów
\usepackage{graphicx}         % Obsługa obrazków
\usepackage{float}            % Lepsze pozycjonowanie tabel/obrazków
\usepackage{url}              % Obsługa linków (dla GitHuba)
\usepackage{hyperref}         % Aktywne linki w PDF
\usepackage{fancyhdr}         % Niestandardowe nagłówki i stopki

% ---------------------------------------------------------
% KONFIGURACJA NAGŁÓWKÓW (WYMÓG ZADANIA)
% ---------------------------------------------------------
\pagestyle{fancy}
\fancyhf{} % Wyczyść domyślne ustawienia
% Lewa strona nagłówka: Tytuł aktualnego rozdziału
\fancyhead[L]{\nouppercase{\leftmark}} 
% Prawa strona nagłówka: Autor (WYMÓG)
\fancyhead[R]{Jan Kowalski} 
% Środek stopki: Numer strony
\fancyfoot[C]{\thepage}

% Ustawienia metadanych PDF (dla ePortalu)
\hypersetup{
    pdftitle={Raport Końcowy},
    pdfauthor={Jan Kowalski}
}

% ---------------------------------------------------------
% DANE DO STRONY TYTUŁOWEJ
% ---------------------------------------------------------
\title{Tytuł Twojego Projektu}
\author{Jan Kowalski} % Podpisany autor
\date{\today}         % Data powstania dokumentu

% =========================================================
% TREŚĆ DOKUMENTU
% =========================================================
\begin{document}

% 1. Strona tytułowa
\maketitle

% 2. Streszczenie (Abstract)
\begin{abstract}
    To jest streszczenie dokumentu. W pracy przedstawiono analizę przykładowego problemu, wykorzystując tabele, wykresy oraz listy. Dokument został sformatowany w systemie \LaTeX{} przy użyciu klasy \texttt{report}.
\end{abstract}

% 3. Spis treści
\tableofcontents

% 4. Rozdziały
\chapter{Wstęp i Teoria}

W tym rozdziale wprowadzamy do tematyki. Poniżej znajduje się przykład listy nienumerowanej oraz numerowanej.

\section{Listy (Wymóg zadania)}

Lista nienumerowana:
\begin{itemize}
    \item Pierwszy ważny punkt analizy.
    \item Drugi punkt dotyczący metodologii.
    \item Trzeci aspekt problemu.
\end{itemize}

Lista numerowana:
\begin{enumerate}
    \item Krok pierwszy: Zbieranie danych.
    \item Krok drugi: Przetwarzanie.
    \item Krok trzeci: Wnioskowanie.
\end{enumerate}

Jak zauważa Nowak w swojej pracy \cite{nowak2023}, listy pomagają porządkować treść.

\chapter{Analiza Danych}

W tym rozdziale prezentujemy wyniki w formie graficznej i tabelarycznej.

\section{Ilustracje}

Poniżej przedstawiono schemat działania systemu (Rysunek \ref{fig:moj_obrazek}).

\begin{figure}[H]
    \centering
    % UWAGA: Musisz mieć plik 'obrazek.jpg' lub 'obrazek.png' w folderze projektu!
    % Jeśli nie masz, zakomentuj linię poniżej znakiem %
    \includegraphics[width=0.6\textwidth]{obrazek.png} 
    \caption{Podpis pod rysunkiem (Wymóg: min. 1 ilustracja)}
    \label{fig:moj_obrazek}
\end{figure}

Zgodnie z badaniami \cite{kowalski2024}, wizualizacja danych przyspiesza ich zrozumienie.

\section{Tabele}

Poniższa tabela (Tabela \ref{tab:dane}) prezentuje zebrane wyniki.

\begin{table}[H]
    \centering
    \caption{Zestawienie wyników pomiarowych (Wymóg: min. 1 tabela)}
    \label{tab:dane}
    \begin{tabular}{|c|l|r|}
        \hline
        \textbf{Lp.} & \textbf{Parametr} & \textbf{Wartość} \\
        \hline
        1 & Temperatura & 23.5 C \\
        \hline
        2 & Ciśnienie & 1013 hPa \\
        \hline
        3 & Wilgotność & 45 \% \\
        \hline
    \end{tabular}
\end{table}

\chapter{Wnioski}

Niniejszy dokument został przygotowany zgodnie z postawionymi wymaganiami.

\section*{Uzasadnienie wyboru klasy dokumentu}
Do wykonania zadania wybrano klasę \textbf{report}. 
Klasa ta jest idealna do tworzenia prac dyplomowych i raportów technicznych średniej długości. 
W przeciwieństwie do klasy \texttt{article}, obsługuje ona podział na rozdziały (\texttt{chapter}) oraz dedykowaną stronę tytułową. 
Z kolei w odróżnieniu od klasy \texttt{book}, klasa \texttt{report} domyślnie nie wymusza druku dwustronnego (nie wstawia pustych stron po rozdziałach), co jest korzystniejsze przy oddawaniu projektu w formie cyfrowej (PDF) na ePortal.

\section*{Dostęp do źródeł}
Źródła całego dokumentu dostępne są w repozytorium pod adresem: \\
\url{https://github.com/twoj-nick/nazwa-repozytorium}

W repozytorium zastosowano plik \texttt{.gitignore}, aby ukryć pliki tymczasowe kompilacji oraz dane osobiste edytora.

% 5. Bibliografia
\begin{thebibliography}{9}
    \bibitem{nowak2023} 
    A. Nowak, \textit{Wstęp do systemu LaTeX}, Wydawnictwo PWr, Wrocław 2023.
    
    \bibitem{kowalski2024} 
    J. Kowalski, \textit{Analiza danych w inżynierii}, O'Reilly, 2024.
\end{thebibliography}

\end{document}