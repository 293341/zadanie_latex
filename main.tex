\documentclass[11pt, a4paper]{report}

\usepackage[polish]{babel}   
\usepackage[utf8]{inputenc} 
\usepackage[T1]{fontenc}     
\usepackage{graphicx}        
\usepackage{float}          
\usepackage{url}           
\usepackage{hyperref}         
\usepackage{fancyhdr}         
\usepackage{booktabs}       

\pagestyle{fancy}
\fancyhf{}
\fancyhead[L]{\nouppercase{\leftmark}} 
\fancyhead[R]{Szymon Płócienniczak} 
\fancyfoot[C]{\thepage}
\renewcommand{\chaptermark}[1]{\markboth{\thechapter.\ #1}{}} 
\fancypagestyle{plain}{
    \fancyhf{} 
    \fancyhead[L]{\nouppercase{\leftmark}} 
    \fancyhead[R]{Szymon Płócienniczak} 
    \fancyfoot[C]{\thepage}
}

\title{Analiza wpływu Generatywnej Sztucznej Inteligencji na efektywność procesów wytwarzania oprogramowania}
\author{Szymon Płócienniczak}
\date{03.12.2025}

\begin{document}

\maketitle

\begin{abstract}
    Niniejszy raport przedstawia analizę wpływu narzędzi opartych na Dużych Modelach Językowych (LLM), takich jak GitHub Copilot czy ChatGPT, na pracę inżynierów oprogramowania. Dokument omawia ewolucję narzędzi, przedstawia dane dotyczące wzrostu produktywności, analizuje zagrożenia bezpieczeństwa oraz wyzwania prawne. Całość uzupełnia prognoza dotycząca autonomicznych agentów AI. Praca została sformatowana w systemie \LaTeX{} z wykorzystaniem klasy \texttt{report}.
\end{abstract}

\tableofcontents

\chapter{Ewolucja narzędzi wspomagających programowanie}

Rozwój inżynierii oprogramowania jest nierozerwalnie związany z automatyzacją. Od prostych autouzupełnień, przez statyczną analizę kodu, aż po systemy oparte na sieciach neuronowych (Transformer) \cite{vaswani2017}.

\section{Rola LLM w cyklu SDLC}
Sztuczna inteligencja staje się integralną częścią środowiska pracy programisty. Jak wskazuje raport GitHub Octoverse \cite{github2023}, adopcja tych narzędzi przyspiesza o 30\% w skali roku.

\subsection{Obszary zastosowań (Lista nienumerowana)}
Narzędzia GenAI wspierają programistów w następujących obszarach:
\begin{itemize}
    \item \textbf{Generowanie kodu boilerplate} - automatyczne tworzenie powtarzalnych fragmentów kodu.
    \item \textbf{Refaktoryzacja} - sugerowanie bardziej optymalnych rozwiązań.
    \item \textbf{Tworzenie testów} - automatyczne generowanie przypadków testowych.
\end{itemize}

\subsection{Etapy adopcji (Lista numerowana)}
Wdrożenie AI w organizacji przebiega w ustalonych fazach:
\begin{enumerate}
    \item \textbf{Eksperymentacja} - indywidualne użycie ChatGPT.
    \item \textbf{Pilotaż} - testy narzędzi typu Copilot for Business.
    \item \textbf{Integracja} - włączenie AI do procesów CI/CD.
    \item \textbf{Pełna adopcja} - AI jako standardowe narzędzie pracy.
\end{enumerate}
\chapter{Analiza wydajności i porównanie modeli}

W tym rozdziale przedstawiono dane ilościowe oraz porównanie modeli.

\section{Wzrost produktywności (Ilustracja)}

Badania Microsoft Research wykazują, że programiści z asystentami AI kończą zadania szybciej. Poniższy wykres (Rysunek \ref{fig:adopcja}) obrazuje ten trend.

\begin{figure}[H]
    \centering
    % UWAGA: Musisz mieć plik wykres.png w folderze!
    \includegraphics[width=0.8\textwidth]{wykres.png} 
    \caption{Wzrost efektywności pracy programistów (Źródło: opracowanie własne)}
    \label{fig:adopcja}
\end{figure}

Seniorzy zyskują głównie na oszczędności czasu przy wyszukiwaniu informacji w dokumentacji \cite{sommerville2015}.

\section{Porównanie modeli (Tabela)}

Tabela \ref{tab:modele} przedstawia zestawienie popularnych rozwiązań.

\begin{table}[H]
    \centering
    \caption{Porównanie modeli LLM (Stan na Q1 2025)}
    \label{tab:modele}
    \renewcommand{\arraystretch}{1.3}
    \begin{tabular}{|l|c|r|l|}
        \hline
        \textbf{Model} & \textbf{Dostawca} & \textbf{Kontekst} & \textbf{Licencja} \\
        \hline
        GPT-4 Turbo & OpenAI & 128k & Komercyjna \\
        \hline
        Claude 3 & Anthropic & 200k & Komercyjna \\
        \hline
        Llama 3 & Meta & 8k & Open Source \\
        \hline
    \end{tabular}
\end{table}
\chapter{Wyzwania etyczne i bezpieczeństwo kodu}

Wdrożenie AI niesie ze sobą ryzyka. Badania Stanford University \cite{stanford2023} sugerują, że kod pisany z AI może zawierać więcej luk bezpieczeństwa.

\section{Główne zagrożenia}
\begin{itemize}
    \item \textbf{Prompt Injection} – manipulacja modelem w celu wyciągnięcia danych.
    \item \textbf{Halucynacje} – sugerowanie nieistniejących bibliotek.
    \item \textbf{Wyciek danych} – przesyłanie firmowego kodu do publicznych modeli.
\end{itemize}

\section{Aspekty prawne}
Kwestia praw autorskich pozostaje nieuregulowana. Mecenas Barta \cite{barta2024} ostrzega przed ryzykiem naruszenia licencji Open Source (np. GPL), gdy model uczył się na chronionym kodzie i "zwraca" go użytkownikowi.
\chapter{Przyszłość: Era Agentów Autonomicznych}

Obecna generacja narzędzi (np. ChatGPT, GitHub Copilot) działa na zasadzie "podpowiadania" (autocomplete). Jest to model pasywny – człowiek musi zainicjować akcję. Jednak branża IT zmierza w kierunku pełnej autonomizacji procesów, co fundamentalnie zmieni rynek pracy.

\section{Era Agentów Autonomicznych}

Rok 2024 i 2025 przynoszą gwałtowny rozwój tzw. agentów AI (przykładem może być projekt Devin czy AutoGPT). Różnica między asystentem a agentem jest fundamentalna i dotyczy pętli decyzyjnej:
\begin{itemize}
    \item \textbf{Asystent:} Czeka na polecenie użytkownika, generuje fragment kodu i kończy działanie. Wymaga ciągłego nadzoru.
    \item \textbf{Agent:} Otrzymuje cel wysokiego poziomu (np. "Stwórz stronę logowania zgodną z OAuth2"), a następnie samodzielnie planuje zadania, pisze kod, uruchamia go, czyta błędy kompilatora i nanosi poprawki aż do skutku.
\end{itemize}

Agenci są zdolni do korzystania z narzędzi zewnętrznych: przeglądarki internetowej, terminala czy klienta SQL, co czyni ich quasi-samodzielnymi pracownikami.

\section{Ewolucja roli programisty: Developer 2.0}

W nadchodzących latach rola programisty ulegnie transformacji. Umiejętność pisania poprawnej składni (syntax) danego języka stanie się drugorzędna wobec umiejętności weryfikacji logiki biznesowej i architektury systemu.

\begin{quote}
    "Programowanie w języku naturalnym stanie się najważniejszym językiem programowania przyszłości. Bariera wejścia do świata tworzenia oprogramowania nigdy nie była niższa." -- Jensen Huang, CEO NVIDIA.
\end{quote}

Przewiduje się, że inżynierowie oprogramowania staną się w większym stopniu "architektami systemów AI" oraz audytorami kodu, niż rzemieślnikami piszącymi linie kodu ręcznie \cite{futuredev2025}. Może to prowadzić do zmniejszenia zapotrzebowania na programistów poziomu Junior, przy jednoczesnym wzroście popytu na ekspertów potrafiących integrować złożone systemy AI.
\chapter{Wnioski i podsumowanie}

Generatywna AI to fundamentalna zmiana w inżynierii oprogramowania, oferująca wzrost produktywności, ale wymagająca nadzoru.

\section*{Uzasadnienie wyboru klasy dokumentu}
Do wykonania zadania wybrano klasę \textbf{report}. Jest ona idealna do raportów technicznych, ponieważ:
\begin{enumerate}
    \item Obsługuje podział na rozdziały (\texttt{chapter}).
    \item Posiada dedykowaną stronę tytułową i streszczenie.
    \item W przeciwieństwie do klasy \texttt{book}, nie wymusza pustych stron (druk dwustronny), co jest lepsze dla formatu cyfrowego PDF.
\end{enumerate}

\section*{Dostęp do źródeł}
Źródła dokumentu dostępne są w repozytorium: \\
\url{https://github.com/293341/zadanie_latex}

Zastosowano plik \texttt{.gitignore} w celu ochrony danych i plików tymczasowych.
\begin{thebibliography}{9}
    \bibitem{nowak2023} 
    A. Nowak, \textit{Wstęp do systemu LaTeX}, Wydawnictwo PWr, Wrocław 2023.
    
    \bibitem{kowalski2024} 
    J. Kowalski, \textit{Analiza danych w inżynierii}, O'Reilly, 2024.
\end{thebibliography}

\end{document}