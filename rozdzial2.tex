\chapter{Analiza wydajności i porównanie modeli}

W tym rozdziale przedstawiono dane ilościowe oraz porównanie modeli.

\section{Wzrost produktywności (Ilustracja)}

Badania Microsoft Research wykazują, że programiści z asystentami AI kończą zadania szybciej. Poniższy wykres (Rysunek \ref{fig:adopcja}) obrazuje ten trend.

\begin{figure}[H]
    \centering
    % UWAGA: Musisz mieć plik wykres.png w folderze!
    \includegraphics[width=0.8\textwidth]{wykres.png} 
    \caption{Wzrost efektywności pracy programistów (Źródło: opracowanie własne)}
    \label{fig:adopcja}
\end{figure}

Seniorzy zyskują głównie na oszczędności czasu przy wyszukiwaniu informacji w dokumentacji \cite{sommerville2015}.

\section{Porównanie modeli (Tabela)}

Tabela \ref{tab:modele} przedstawia zestawienie popularnych rozwiązań.

\begin{table}[H]
    \centering
    \caption{Porównanie modeli LLM (Stan na Q1 2025)}
    \label{tab:modele}
    \renewcommand{\arraystretch}{1.3}
    \begin{tabular}{|l|c|r|l|}
        \hline
        \textbf{Model} & \textbf{Dostawca} & \textbf{Kontekst} & \textbf{Licencja} \\
        \hline
        GPT-4 Turbo & OpenAI & 128k & Komercyjna \\
        \hline
        Claude 3 & Anthropic & 200k & Komercyjna \\
        \hline
        Llama 3 & Meta & 8k & Open Source \\
        \hline
    \end{tabular}
\end{table}