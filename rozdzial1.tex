\chapter{Ewolucja narzędzi wspomagających programowanie}

Rozwój inżynierii oprogramowania jest nierozerwalnie związany z automatyzacją. Od prostych autouzupełnień, przez statyczną analizę kodu, aż po systemy oparte na sieciach neuronowych (Transformer) \cite{vaswani2017}.

\section{Rola LLM w cyklu SDLC}
Sztuczna inteligencja staje się integralną częścią środowiska pracy programisty. Jak wskazuje raport GitHub Octoverse \cite{github2023}, adopcja tych narzędzi przyspiesza o 30\% w skali roku.

\subsection{Obszary zastosowań (Lista nienumerowana)}
Narzędzia GenAI wspierają programistów w następujących obszarach:
\begin{itemize}
    \item \textbf{Generowanie kodu boilerplate} - automatyczne tworzenie powtarzalnych fragmentów kodu.
    \item \textbf{Refaktoryzacja} - sugerowanie bardziej optymalnych rozwiązań.
    \item \textbf{Tworzenie testów} - automatyczne generowanie przypadków testowych.
\end{itemize}

\subsection{Etapy adopcji (Lista numerowana)}
Wdrożenie AI w organizacji przebiega w ustalonych fazach:
\begin{enumerate}
    \item \textbf{Eksperymentacja} - indywidualne użycie ChatGPT.
    \item \textbf{Pilotaż} - testy narzędzi typu Copilot for Business.
    \item \textbf{Integracja} - włączenie AI do procesów CI/CD.
    \item \textbf{Pełna adopcja} - AI jako standardowe narzędzie pracy.
\end{enumerate}