\chapter{Wyzwania etyczne i bezpieczeństwo kodu}

Masowe wdrożenie Generatywnej Sztucznej Inteligencji niesie ze sobą nie tylko korzyści ekonomiczne, ale również istotne ryzyka techniczne i prawne. W tym rozdziale omówiono problemy związane z jakością generowanego kodu, podatnościami na ataki oraz niejasną sytuacją prawną własności intelektualnej.

\section{Podatności i błędy bezpieczeństwa}

Badania przeprowadzone przez zespół badawczy uniwersytetu Stanford wykazały niepokojący trend: programiści korzystający z asystentów AI mają tendencję do pisania mniej bezpiecznego kodu, będąc jednocześnie bardziej pewnymi jego poprawności \cite{stanford2023}. Wynika to z faktu, że modele LLM są trenowane na kodzie z internetu, który często zawiera błędy.

Główne kategorie zagrożeń obejmują:
\begin{itemize}
    \item \textbf{Ataki typu Prompt Injection} – manipulacja modelem poprzez odpowiednio spreparowane instrukcje w celu wyciągnięcia poufnych danych z pamięci kontekstowej modelu.
    \item \textbf{Halucynacje biblioteczne (Package Hallucinations)} – modele potrafią sugerować użycie nieistniejących pakietów o wiarygodnie brzmiących nazwach. Hakerzy wykorzystują to zjawisko, publikując złośliwe pakiety o tych właśnie nazwach (atak typu \textit{Dependency Confusion}).
    \item \textbf{Hardcoding danych uwierzytelniających} – AI często generuje przykładowy kod zawierający "sztywne" hasła, klucze API lub adresy serwerów, które niedoświadczeni programiści mogą nieświadomie skopiować na środowisko produkcyjne.
\end{itemize}

\section{Aspekty prawne i prawa autorskie}

Kwestia praw autorskich do kodu wygenerowanego przez AI pozostaje nieuregulowana w wielu jurysdykcjach, w tym w Unii Europejskiej i USA. Jak zauważa mecenas Barta w analizie dla branży IT \cite{barta2024}, pojawia się ryzyko tzw. "prania licencji". 

Model uczony na kodzie objętym restrykcyjną licencją GPL może wygenerować fragment kodu łudząco podobny do oryginału, ale pozbawiony informacji o licencji. Wykorzystanie takiego fragmentu w komercyjnym oprogramowaniu zamkniętym (proprietary) może narazić firmę na poważne konsekwencje prawne i konieczność otwarcia kodu źródłowego całego projektu.