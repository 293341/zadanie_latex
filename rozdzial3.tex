\chapter{Wyzwania etyczne i bezpieczeństwo kodu}

Wdrożenie AI niesie ze sobą ryzyka. Badania Stanford University \cite{stanford2023} sugerują, że kod pisany z AI może zawierać więcej luk bezpieczeństwa.

\section{Główne zagrożenia}
\begin{itemize}
    \item \textbf{Prompt Injection} – manipulacja modelem w celu wyciągnięcia danych.
    \item \textbf{Halucynacje} – sugerowanie nieistniejących bibliotek.
    \item \textbf{Wyciek danych} – przesyłanie firmowego kodu do publicznych modeli.
\end{itemize}

\section{Aspekty prawne}
Kwestia praw autorskich pozostaje nieuregulowana. Mecenas Barta \cite{barta2024} ostrzega przed ryzykiem naruszenia licencji Open Source (np. GPL), gdy model uczył się na chronionym kodzie i "zwraca" go użytkownikowi.