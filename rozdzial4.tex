\chapter{Przyszłość: Era Agentów Autonomicznych}

Obecna generacja narzędzi (np. ChatGPT, GitHub Copilot) działa na zasadzie "podpowiadania" (autocomplete). Jest to model pasywny – człowiek musi zainicjować akcję. Jednak branża IT zmierza w kierunku pełnej autonomizacji procesów, co fundamentalnie zmieni rynek pracy.

\section{Era Agentów Autonomicznych}

Rok 2024 i 2025 przynoszą gwałtowny rozwój tzw. agentów AI (przykładem może być projekt Devin czy AutoGPT). Różnica między asystentem a agentem jest fundamentalna i dotyczy pętli decyzyjnej:
\begin{itemize}
    \item \textbf{Asystent:} Czeka na polecenie użytkownika, generuje fragment kodu i kończy działanie. Wymaga ciągłego nadzoru.
    \item \textbf{Agent:} Otrzymuje cel wysokiego poziomu (np. "Stwórz stronę logowania zgodną z OAuth2"), a następnie samodzielnie planuje zadania, pisze kod, uruchamia go, czyta błędy kompilatora i nanosi poprawki aż do skutku.
\end{itemize}

Agenci są zdolni do korzystania z narzędzi zewnętrznych: przeglądarki internetowej, terminala czy klienta SQL, co czyni ich quasi-samodzielnymi pracownikami.

\section{Ewolucja roli programisty: Developer 2.0}

W nadchodzących latach rola programisty ulegnie transformacji. Umiejętność pisania poprawnej składni (syntax) danego języka stanie się drugorzędna wobec umiejętności weryfikacji logiki biznesowej i architektury systemu.

\begin{quote}
    "Programowanie w języku naturalnym stanie się najważniejszym językiem programowania przyszłości. Bariera wejścia do świata tworzenia oprogramowania nigdy nie była niższa." -- Jensen Huang, CEO NVIDIA.
\end{quote}

Przewiduje się, że inżynierowie oprogramowania staną się w większym stopniu "architektami systemów AI" oraz audytorami kodu, niż rzemieślnikami piszącymi linie kodu ręcznie \cite{futuredev2025}. Może to prowadzić do zmniejszenia zapotrzebowania na programistów poziomu Junior, przy jednoczesnym wzroście popytu na ekspertów potrafiących integrować złożone systemy AI.